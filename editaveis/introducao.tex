\chapter[Introdução]{Introdução}

O primeiro registro de cadeiras de rodas, foi uma inscrição encontrada em uma pedra ardósia na China e uma cama de criança descrita em um friso em um vaso grego, tanto que remonta ao século 5~AC~\cite{joseph:puttingthewhee:2012}, \cite{maggie:whoinvented:2012}. Observa-se que desde tempos ancestrais existe a preocupação com a melhoria da qualidade de vida, tal como inserir pessoas na sociedade e no cotidiano. 

Segundo uma pesquisa realizada em 2010 pelo IBGE, existem cerca de 4,4 milhões de indivíduos incapazes ou com grandes dificuldades de locomoção em todo o Brasil~\cite{ibge:cartilha:2010}, cadeirantes, em sua maioria. A utilização de cadeiras de rodas realmente torna o deficiente físico muito mais independente, porém, a distância entre a independência e a verdadeira mobilidade é muito grande, o uso da cadeira de rodas manual pode ser adequada para jovens em forma, mas não para cadeirantes com sobrepeso, com os braços enfraquecidos ou idosos. Mesmo em locais controlados, o uso da cadeira de rodas manual requer a algumas pessoas um assistente, a cadeira de rodas elétrica dá ao cadeirante o conforto e a mobilidade necessárias em qualquer ambiente minimamente acessível, no entanto, cadeiras de rodas elétricas tem um alto custo e muitas vezes são pouco portáteis.

Visto essa dificuldade, locais como shoppings e supermercados oferecem gratuitamente cadeiras de rodas elétricas para uso interno, tal iniciativa traz melhorias na qualidade de vida desse indivíduo, porém, na maioria dos casos, essas cadeiras de rodas genéricas não se adequam ao corpo do cadeirante, ou seja, são necessárias outras alternativas para combater essa dificuldade. Projetos de automação de cadeiras de rodas manuais~\cite{brunel:wheelchair:2004}, \cite{artigo_rudi}, \cite{patent_cadeira_rodas_eletrica},	 \cite{marcos:controle:2002} dão ao cadeirante a facilidade de mobilidade sem a ajuda de outras pessoas e ao mesmo tempo utilizam-se do fato de que a cadeira de rodas manual do cadeirante já está totalmente adaptada a ele.

A proposta desse trabalho é desenvolver um kit universal de automação de cadeira de rodas, o que tornará possível descartar o uso do trabalho humano para locomovê-la. Para garantir que esse produto seja compatível com o maior número de cadeiras de rodas do mercado possível, seguir-se-á o padrão especificado pela NBR 9050 (ABNT, 2004) e as dimensões do INMETRO~\cite{inmetro}. Cabe ressaltar que o produto deste trabalho consiste no conjunto que oferecerá autonomia suficiente para o descarte de trabalho físico do homem para locomover a cadeira de locais controlados. Em princípio, a ideia consiste em uma aplicação desse novo sistema de cadeira de rodas automatizada a fim de atender cadeirantes de shopping centers. Portanto, para o cadeirante conduzir a cadeira de rodas de maneira eficiente e segura, o dispositivo dispõe de um controle via joystick, com opção para controle via celular (Bluetooth~\cite{bluetooth}). Para dar uma imagem ao produto, pensa-se em um dispositivo semelhante a uma mala de viagem que será acoplado à cadeira de rodas. Dentro dessa mala conterão os motores, baterias, dispositivos de controle e um comando do tipo joystick que será preso à cadeira de rodas, próximo ao cadeirante.

O primeiro objetivo desse trabalho é o levantamento teórico das tecnologias já existentes no ramo de cadeiras de rodas motorizadas, a fim de adequar o projeto as melhores opções. O projeto está dividido em quatro ramos: power train, estrutura, controle e interface com o usuário, pretende-se ter envolvimento de todas as engenharias (Automotiva, Eletrônica, Energia e Software) nos quatro ramos durante toda a execução do projeto.

Tem-se como objetivo final o desenvolvimento de um kit universal de automação de cadeiras de rodas manuais portátil e removível, este consiste em um sistema que quando acoplado a cadeiras de rodas manuais convencionais, transforme-as em cadeiras de rodas motorizadas. Tendo como público alvo os shopping centers, por tal razão o produto deve funcionar em um ambiente controlado (plano sem imperfeições no solo). Oferecendo um custo significantemente reduzido quando comparado a uma cadeira de rodas motorizada e, além disso, a possibilidade de o cadeirante usufruir dos benefícios da cadeira motorizada sem perder a liberdade e ergonomia que o seu modelo manual proporciona.

