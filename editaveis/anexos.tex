\begin{anexosenv}

\partanexos

\chapter{Formulários}
	\section{Estórias de Usuário}
	\label{sec:form_us}

		\begin{citacao}

			\begin{itemize}
				\item Projeto: nesse campo os analistas informarão o projeto ao qual a user story se refere. Esse campo ajudará a organização e seleção das user story no projeto em desenvolvimento.
				\item Autor: refere-se à pessoa responsável por descrever a user story. Com essa identificação, o time saberá a quem recorrer no momento em que ocorrer algum problema ou surgir alguma dúvida durante a implementação do sistema.
				\item Título: nesse campo, os analistas informarão como a user story será chamada e identificada. Esse título deve ser sugestivo referindo-se ao conteúdo da user story.
				\item Descrição: nesse local, todas as informações sobre o sistema deverão ser descritas: o entendimento dos stakeholders sobre o sistema, as regras de negócio, validação de campos, o que o sistema deverá realizar, o que será permitido ou não pelo sistema, o que o usuário poderá realizar em seu acesso. Quanto maior o número de informações aqui relatadas mais rica será a user story, maior o entendimento do time.
			\end{itemize}

		\end{citacao}

	\section{QaSe}
	\label{sec:form_qase}

		\begin{citacao}
			\begin{itemize}

				\item Projeto: nesse campo o analista informará qual o projeto ao qual o scenario pertence. Esse campo ajudará a organização e seleção das user stories no projeto em desenvolvimento.
				\item Autor: refere-se ao analista de requisitos responsável pela modelagem do quality scenario. Com essa identificação, o time saberá a quem recorrer no momento em que ocorrer algum problema ou surgir alguma dúvida durante a implementação do sistema. O analista atuará no suporte ao time e o acompanhará durante todo o desenvolvimento.
				\item Título: nesse campo será informado o nome do quality scenario. O título deve ser sugestivo referindo-se ao requisito não-funcional que será abordado no scenario.
				\item Descrição: nesse local, todas as informações relacionadas às necessidades solicitadas referentes ao requisito não-funcional: as situações nas quais os NFR89 serão abordados no sistema, os aspectos comportamentais, as ações do sistema para atender ao requisito, as condições de uso, deverão ser descritas. Nesse campo, uma narrativa explicando o negócio ao usuário e ao time deve ser escrita, utilizando uma linguagem natural, simples e objetiva. Quanto maior a quantidade e qualidade das informações, mais compreensível se tornará o requisito não-funcional.
				\item Interdependências entre NFR: o analista informará quais os NFR que possuem alguma dependência em relação ao requisito modelado no scenario e com quais requisitos o mesmo possui dependência. Descreverá como são as dependências entre os requisitos não-funcionais.
				\item Conflitos entre NFR: quando há dependências entre requisitos que de alguma forma interferem negativamente no requisito não-funcional modelado, gera algum tipo de conflito no sistema que precisa ser descrito. Os conflitos são identificados e analisados, para que seja definido pelo time como serão tratados.
				\item Impactos do não atendimento ao NFR: o requisito não-funcional modelado no documento de quality scenarios precisa ser atendido, caso não seja, alguns impactos poderão afetar o sistema. Esses impactos precisam ser informados ao time por meio desse campo, a fim de confirmar a importância daquele requisito
				não-funcional para o sistema.

			\end{itemize}

		\end{citacao}

\chapter{Segundo Anexo}

Texto do segundo anexo.

\end{anexosenv}
